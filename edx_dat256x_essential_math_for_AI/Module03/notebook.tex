
% Default to the notebook output style

    


% Inherit from the specified cell style.




    
\documentclass[11pt]{article}

    
    
    \usepackage[T1]{fontenc}
    % Nicer default font (+ math font) than Computer Modern for most use cases
    \usepackage{mathpazo}

    % Basic figure setup, for now with no caption control since it's done
    % automatically by Pandoc (which extracts ![](path) syntax from Markdown).
    \usepackage{graphicx}
    % We will generate all images so they have a width \maxwidth. This means
    % that they will get their normal width if they fit onto the page, but
    % are scaled down if they would overflow the margins.
    \makeatletter
    \def\maxwidth{\ifdim\Gin@nat@width>\linewidth\linewidth
    \else\Gin@nat@width\fi}
    \makeatother
    \let\Oldincludegraphics\includegraphics
    % Set max figure width to be 80% of text width, for now hardcoded.
    \renewcommand{\includegraphics}[1]{\Oldincludegraphics[width=.8\maxwidth]{#1}}
    % Ensure that by default, figures have no caption (until we provide a
    % proper Figure object with a Caption API and a way to capture that
    % in the conversion process - todo).
    \usepackage{caption}
    \DeclareCaptionLabelFormat{nolabel}{}
    \captionsetup{labelformat=nolabel}

    \usepackage{adjustbox} % Used to constrain images to a maximum size 
    \usepackage{xcolor} % Allow colors to be defined
    \usepackage{enumerate} % Needed for markdown enumerations to work
    \usepackage{geometry} % Used to adjust the document margins
    \usepackage{amsmath} % Equations
    \usepackage{amssymb} % Equations
    \usepackage{textcomp} % defines textquotesingle
    % Hack from http://tex.stackexchange.com/a/47451/13684:
    \AtBeginDocument{%
        \def\PYZsq{\textquotesingle}% Upright quotes in Pygmentized code
    }
    \usepackage{upquote} % Upright quotes for verbatim code
    \usepackage{eurosym} % defines \euro
    \usepackage[mathletters]{ucs} % Extended unicode (utf-8) support
    \usepackage[utf8x]{inputenc} % Allow utf-8 characters in the tex document
    \usepackage{fancyvrb} % verbatim replacement that allows latex
    \usepackage{grffile} % extends the file name processing of package graphics 
                         % to support a larger range 
    % The hyperref package gives us a pdf with properly built
    % internal navigation ('pdf bookmarks' for the table of contents,
    % internal cross-reference links, web links for URLs, etc.)
    \usepackage{hyperref}
    \usepackage{longtable} % longtable support required by pandoc >1.10
    \usepackage{booktabs}  % table support for pandoc > 1.12.2
    \usepackage[inline]{enumitem} % IRkernel/repr support (it uses the enumerate* environment)
    \usepackage[normalem]{ulem} % ulem is needed to support strikethroughs (\sout)
                                % normalem makes italics be italics, not underlines
    

    
    
    % Colors for the hyperref package
    \definecolor{urlcolor}{rgb}{0,.145,.698}
    \definecolor{linkcolor}{rgb}{.71,0.21,0.01}
    \definecolor{citecolor}{rgb}{.12,.54,.11}

    % ANSI colors
    \definecolor{ansi-black}{HTML}{3E424D}
    \definecolor{ansi-black-intense}{HTML}{282C36}
    \definecolor{ansi-red}{HTML}{E75C58}
    \definecolor{ansi-red-intense}{HTML}{B22B31}
    \definecolor{ansi-green}{HTML}{00A250}
    \definecolor{ansi-green-intense}{HTML}{007427}
    \definecolor{ansi-yellow}{HTML}{DDB62B}
    \definecolor{ansi-yellow-intense}{HTML}{B27D12}
    \definecolor{ansi-blue}{HTML}{208FFB}
    \definecolor{ansi-blue-intense}{HTML}{0065CA}
    \definecolor{ansi-magenta}{HTML}{D160C4}
    \definecolor{ansi-magenta-intense}{HTML}{A03196}
    \definecolor{ansi-cyan}{HTML}{60C6C8}
    \definecolor{ansi-cyan-intense}{HTML}{258F8F}
    \definecolor{ansi-white}{HTML}{C5C1B4}
    \definecolor{ansi-white-intense}{HTML}{A1A6B2}

    % commands and environments needed by pandoc snippets
    % extracted from the output of `pandoc -s`
    \providecommand{\tightlist}{%
      \setlength{\itemsep}{0pt}\setlength{\parskip}{0pt}}
    \DefineVerbatimEnvironment{Highlighting}{Verbatim}{commandchars=\\\{\}}
    % Add ',fontsize=\small' for more characters per line
    \newenvironment{Shaded}{}{}
    \newcommand{\KeywordTok}[1]{\textcolor[rgb]{0.00,0.44,0.13}{\textbf{{#1}}}}
    \newcommand{\DataTypeTok}[1]{\textcolor[rgb]{0.56,0.13,0.00}{{#1}}}
    \newcommand{\DecValTok}[1]{\textcolor[rgb]{0.25,0.63,0.44}{{#1}}}
    \newcommand{\BaseNTok}[1]{\textcolor[rgb]{0.25,0.63,0.44}{{#1}}}
    \newcommand{\FloatTok}[1]{\textcolor[rgb]{0.25,0.63,0.44}{{#1}}}
    \newcommand{\CharTok}[1]{\textcolor[rgb]{0.25,0.44,0.63}{{#1}}}
    \newcommand{\StringTok}[1]{\textcolor[rgb]{0.25,0.44,0.63}{{#1}}}
    \newcommand{\CommentTok}[1]{\textcolor[rgb]{0.38,0.63,0.69}{\textit{{#1}}}}
    \newcommand{\OtherTok}[1]{\textcolor[rgb]{0.00,0.44,0.13}{{#1}}}
    \newcommand{\AlertTok}[1]{\textcolor[rgb]{1.00,0.00,0.00}{\textbf{{#1}}}}
    \newcommand{\FunctionTok}[1]{\textcolor[rgb]{0.02,0.16,0.49}{{#1}}}
    \newcommand{\RegionMarkerTok}[1]{{#1}}
    \newcommand{\ErrorTok}[1]{\textcolor[rgb]{1.00,0.00,0.00}{\textbf{{#1}}}}
    \newcommand{\NormalTok}[1]{{#1}}
    
    % Additional commands for more recent versions of Pandoc
    \newcommand{\ConstantTok}[1]{\textcolor[rgb]{0.53,0.00,0.00}{{#1}}}
    \newcommand{\SpecialCharTok}[1]{\textcolor[rgb]{0.25,0.44,0.63}{{#1}}}
    \newcommand{\VerbatimStringTok}[1]{\textcolor[rgb]{0.25,0.44,0.63}{{#1}}}
    \newcommand{\SpecialStringTok}[1]{\textcolor[rgb]{0.73,0.40,0.53}{{#1}}}
    \newcommand{\ImportTok}[1]{{#1}}
    \newcommand{\DocumentationTok}[1]{\textcolor[rgb]{0.73,0.13,0.13}{\textit{{#1}}}}
    \newcommand{\AnnotationTok}[1]{\textcolor[rgb]{0.38,0.63,0.69}{\textbf{\textit{{#1}}}}}
    \newcommand{\CommentVarTok}[1]{\textcolor[rgb]{0.38,0.63,0.69}{\textbf{\textit{{#1}}}}}
    \newcommand{\VariableTok}[1]{\textcolor[rgb]{0.10,0.09,0.49}{{#1}}}
    \newcommand{\ControlFlowTok}[1]{\textcolor[rgb]{0.00,0.44,0.13}{\textbf{{#1}}}}
    \newcommand{\OperatorTok}[1]{\textcolor[rgb]{0.40,0.40,0.40}{{#1}}}
    \newcommand{\BuiltInTok}[1]{{#1}}
    \newcommand{\ExtensionTok}[1]{{#1}}
    \newcommand{\PreprocessorTok}[1]{\textcolor[rgb]{0.74,0.48,0.00}{{#1}}}
    \newcommand{\AttributeTok}[1]{\textcolor[rgb]{0.49,0.56,0.16}{{#1}}}
    \newcommand{\InformationTok}[1]{\textcolor[rgb]{0.38,0.63,0.69}{\textbf{\textit{{#1}}}}}
    \newcommand{\WarningTok}[1]{\textcolor[rgb]{0.38,0.63,0.69}{\textbf{\textit{{#1}}}}}
    
    
    % Define a nice break command that doesn't care if a line doesn't already
    % exist.
    \def\br{\hspace*{\fill} \\* }
    % Math Jax compatability definitions
    \def\gt{>}
    \def\lt{<}
    % Document parameters
    \title{03-04-More Matrices}
    
    
    

    % Pygments definitions
    
\makeatletter
\def\PY@reset{\let\PY@it=\relax \let\PY@bf=\relax%
    \let\PY@ul=\relax \let\PY@tc=\relax%
    \let\PY@bc=\relax \let\PY@ff=\relax}
\def\PY@tok#1{\csname PY@tok@#1\endcsname}
\def\PY@toks#1+{\ifx\relax#1\empty\else%
    \PY@tok{#1}\expandafter\PY@toks\fi}
\def\PY@do#1{\PY@bc{\PY@tc{\PY@ul{%
    \PY@it{\PY@bf{\PY@ff{#1}}}}}}}
\def\PY#1#2{\PY@reset\PY@toks#1+\relax+\PY@do{#2}}

\expandafter\def\csname PY@tok@w\endcsname{\def\PY@tc##1{\textcolor[rgb]{0.73,0.73,0.73}{##1}}}
\expandafter\def\csname PY@tok@c\endcsname{\let\PY@it=\textit\def\PY@tc##1{\textcolor[rgb]{0.25,0.50,0.50}{##1}}}
\expandafter\def\csname PY@tok@cp\endcsname{\def\PY@tc##1{\textcolor[rgb]{0.74,0.48,0.00}{##1}}}
\expandafter\def\csname PY@tok@k\endcsname{\let\PY@bf=\textbf\def\PY@tc##1{\textcolor[rgb]{0.00,0.50,0.00}{##1}}}
\expandafter\def\csname PY@tok@kp\endcsname{\def\PY@tc##1{\textcolor[rgb]{0.00,0.50,0.00}{##1}}}
\expandafter\def\csname PY@tok@kt\endcsname{\def\PY@tc##1{\textcolor[rgb]{0.69,0.00,0.25}{##1}}}
\expandafter\def\csname PY@tok@o\endcsname{\def\PY@tc##1{\textcolor[rgb]{0.40,0.40,0.40}{##1}}}
\expandafter\def\csname PY@tok@ow\endcsname{\let\PY@bf=\textbf\def\PY@tc##1{\textcolor[rgb]{0.67,0.13,1.00}{##1}}}
\expandafter\def\csname PY@tok@nb\endcsname{\def\PY@tc##1{\textcolor[rgb]{0.00,0.50,0.00}{##1}}}
\expandafter\def\csname PY@tok@nf\endcsname{\def\PY@tc##1{\textcolor[rgb]{0.00,0.00,1.00}{##1}}}
\expandafter\def\csname PY@tok@nc\endcsname{\let\PY@bf=\textbf\def\PY@tc##1{\textcolor[rgb]{0.00,0.00,1.00}{##1}}}
\expandafter\def\csname PY@tok@nn\endcsname{\let\PY@bf=\textbf\def\PY@tc##1{\textcolor[rgb]{0.00,0.00,1.00}{##1}}}
\expandafter\def\csname PY@tok@ne\endcsname{\let\PY@bf=\textbf\def\PY@tc##1{\textcolor[rgb]{0.82,0.25,0.23}{##1}}}
\expandafter\def\csname PY@tok@nv\endcsname{\def\PY@tc##1{\textcolor[rgb]{0.10,0.09,0.49}{##1}}}
\expandafter\def\csname PY@tok@no\endcsname{\def\PY@tc##1{\textcolor[rgb]{0.53,0.00,0.00}{##1}}}
\expandafter\def\csname PY@tok@nl\endcsname{\def\PY@tc##1{\textcolor[rgb]{0.63,0.63,0.00}{##1}}}
\expandafter\def\csname PY@tok@ni\endcsname{\let\PY@bf=\textbf\def\PY@tc##1{\textcolor[rgb]{0.60,0.60,0.60}{##1}}}
\expandafter\def\csname PY@tok@na\endcsname{\def\PY@tc##1{\textcolor[rgb]{0.49,0.56,0.16}{##1}}}
\expandafter\def\csname PY@tok@nt\endcsname{\let\PY@bf=\textbf\def\PY@tc##1{\textcolor[rgb]{0.00,0.50,0.00}{##1}}}
\expandafter\def\csname PY@tok@nd\endcsname{\def\PY@tc##1{\textcolor[rgb]{0.67,0.13,1.00}{##1}}}
\expandafter\def\csname PY@tok@s\endcsname{\def\PY@tc##1{\textcolor[rgb]{0.73,0.13,0.13}{##1}}}
\expandafter\def\csname PY@tok@sd\endcsname{\let\PY@it=\textit\def\PY@tc##1{\textcolor[rgb]{0.73,0.13,0.13}{##1}}}
\expandafter\def\csname PY@tok@si\endcsname{\let\PY@bf=\textbf\def\PY@tc##1{\textcolor[rgb]{0.73,0.40,0.53}{##1}}}
\expandafter\def\csname PY@tok@se\endcsname{\let\PY@bf=\textbf\def\PY@tc##1{\textcolor[rgb]{0.73,0.40,0.13}{##1}}}
\expandafter\def\csname PY@tok@sr\endcsname{\def\PY@tc##1{\textcolor[rgb]{0.73,0.40,0.53}{##1}}}
\expandafter\def\csname PY@tok@ss\endcsname{\def\PY@tc##1{\textcolor[rgb]{0.10,0.09,0.49}{##1}}}
\expandafter\def\csname PY@tok@sx\endcsname{\def\PY@tc##1{\textcolor[rgb]{0.00,0.50,0.00}{##1}}}
\expandafter\def\csname PY@tok@m\endcsname{\def\PY@tc##1{\textcolor[rgb]{0.40,0.40,0.40}{##1}}}
\expandafter\def\csname PY@tok@gh\endcsname{\let\PY@bf=\textbf\def\PY@tc##1{\textcolor[rgb]{0.00,0.00,0.50}{##1}}}
\expandafter\def\csname PY@tok@gu\endcsname{\let\PY@bf=\textbf\def\PY@tc##1{\textcolor[rgb]{0.50,0.00,0.50}{##1}}}
\expandafter\def\csname PY@tok@gd\endcsname{\def\PY@tc##1{\textcolor[rgb]{0.63,0.00,0.00}{##1}}}
\expandafter\def\csname PY@tok@gi\endcsname{\def\PY@tc##1{\textcolor[rgb]{0.00,0.63,0.00}{##1}}}
\expandafter\def\csname PY@tok@gr\endcsname{\def\PY@tc##1{\textcolor[rgb]{1.00,0.00,0.00}{##1}}}
\expandafter\def\csname PY@tok@ge\endcsname{\let\PY@it=\textit}
\expandafter\def\csname PY@tok@gs\endcsname{\let\PY@bf=\textbf}
\expandafter\def\csname PY@tok@gp\endcsname{\let\PY@bf=\textbf\def\PY@tc##1{\textcolor[rgb]{0.00,0.00,0.50}{##1}}}
\expandafter\def\csname PY@tok@go\endcsname{\def\PY@tc##1{\textcolor[rgb]{0.53,0.53,0.53}{##1}}}
\expandafter\def\csname PY@tok@gt\endcsname{\def\PY@tc##1{\textcolor[rgb]{0.00,0.27,0.87}{##1}}}
\expandafter\def\csname PY@tok@err\endcsname{\def\PY@bc##1{\setlength{\fboxsep}{0pt}\fcolorbox[rgb]{1.00,0.00,0.00}{1,1,1}{\strut ##1}}}
\expandafter\def\csname PY@tok@kc\endcsname{\let\PY@bf=\textbf\def\PY@tc##1{\textcolor[rgb]{0.00,0.50,0.00}{##1}}}
\expandafter\def\csname PY@tok@kd\endcsname{\let\PY@bf=\textbf\def\PY@tc##1{\textcolor[rgb]{0.00,0.50,0.00}{##1}}}
\expandafter\def\csname PY@tok@kn\endcsname{\let\PY@bf=\textbf\def\PY@tc##1{\textcolor[rgb]{0.00,0.50,0.00}{##1}}}
\expandafter\def\csname PY@tok@kr\endcsname{\let\PY@bf=\textbf\def\PY@tc##1{\textcolor[rgb]{0.00,0.50,0.00}{##1}}}
\expandafter\def\csname PY@tok@bp\endcsname{\def\PY@tc##1{\textcolor[rgb]{0.00,0.50,0.00}{##1}}}
\expandafter\def\csname PY@tok@fm\endcsname{\def\PY@tc##1{\textcolor[rgb]{0.00,0.00,1.00}{##1}}}
\expandafter\def\csname PY@tok@vc\endcsname{\def\PY@tc##1{\textcolor[rgb]{0.10,0.09,0.49}{##1}}}
\expandafter\def\csname PY@tok@vg\endcsname{\def\PY@tc##1{\textcolor[rgb]{0.10,0.09,0.49}{##1}}}
\expandafter\def\csname PY@tok@vi\endcsname{\def\PY@tc##1{\textcolor[rgb]{0.10,0.09,0.49}{##1}}}
\expandafter\def\csname PY@tok@vm\endcsname{\def\PY@tc##1{\textcolor[rgb]{0.10,0.09,0.49}{##1}}}
\expandafter\def\csname PY@tok@sa\endcsname{\def\PY@tc##1{\textcolor[rgb]{0.73,0.13,0.13}{##1}}}
\expandafter\def\csname PY@tok@sb\endcsname{\def\PY@tc##1{\textcolor[rgb]{0.73,0.13,0.13}{##1}}}
\expandafter\def\csname PY@tok@sc\endcsname{\def\PY@tc##1{\textcolor[rgb]{0.73,0.13,0.13}{##1}}}
\expandafter\def\csname PY@tok@dl\endcsname{\def\PY@tc##1{\textcolor[rgb]{0.73,0.13,0.13}{##1}}}
\expandafter\def\csname PY@tok@s2\endcsname{\def\PY@tc##1{\textcolor[rgb]{0.73,0.13,0.13}{##1}}}
\expandafter\def\csname PY@tok@sh\endcsname{\def\PY@tc##1{\textcolor[rgb]{0.73,0.13,0.13}{##1}}}
\expandafter\def\csname PY@tok@s1\endcsname{\def\PY@tc##1{\textcolor[rgb]{0.73,0.13,0.13}{##1}}}
\expandafter\def\csname PY@tok@mb\endcsname{\def\PY@tc##1{\textcolor[rgb]{0.40,0.40,0.40}{##1}}}
\expandafter\def\csname PY@tok@mf\endcsname{\def\PY@tc##1{\textcolor[rgb]{0.40,0.40,0.40}{##1}}}
\expandafter\def\csname PY@tok@mh\endcsname{\def\PY@tc##1{\textcolor[rgb]{0.40,0.40,0.40}{##1}}}
\expandafter\def\csname PY@tok@mi\endcsname{\def\PY@tc##1{\textcolor[rgb]{0.40,0.40,0.40}{##1}}}
\expandafter\def\csname PY@tok@il\endcsname{\def\PY@tc##1{\textcolor[rgb]{0.40,0.40,0.40}{##1}}}
\expandafter\def\csname PY@tok@mo\endcsname{\def\PY@tc##1{\textcolor[rgb]{0.40,0.40,0.40}{##1}}}
\expandafter\def\csname PY@tok@ch\endcsname{\let\PY@it=\textit\def\PY@tc##1{\textcolor[rgb]{0.25,0.50,0.50}{##1}}}
\expandafter\def\csname PY@tok@cm\endcsname{\let\PY@it=\textit\def\PY@tc##1{\textcolor[rgb]{0.25,0.50,0.50}{##1}}}
\expandafter\def\csname PY@tok@cpf\endcsname{\let\PY@it=\textit\def\PY@tc##1{\textcolor[rgb]{0.25,0.50,0.50}{##1}}}
\expandafter\def\csname PY@tok@c1\endcsname{\let\PY@it=\textit\def\PY@tc##1{\textcolor[rgb]{0.25,0.50,0.50}{##1}}}
\expandafter\def\csname PY@tok@cs\endcsname{\let\PY@it=\textit\def\PY@tc##1{\textcolor[rgb]{0.25,0.50,0.50}{##1}}}

\def\PYZbs{\char`\\}
\def\PYZus{\char`\_}
\def\PYZob{\char`\{}
\def\PYZcb{\char`\}}
\def\PYZca{\char`\^}
\def\PYZam{\char`\&}
\def\PYZlt{\char`\<}
\def\PYZgt{\char`\>}
\def\PYZsh{\char`\#}
\def\PYZpc{\char`\%}
\def\PYZdl{\char`\$}
\def\PYZhy{\char`\-}
\def\PYZsq{\char`\'}
\def\PYZdq{\char`\"}
\def\PYZti{\char`\~}
% for compatibility with earlier versions
\def\PYZat{@}
\def\PYZlb{[}
\def\PYZrb{]}
\makeatother


    % Exact colors from NB
    \definecolor{incolor}{rgb}{0.0, 0.0, 0.5}
    \definecolor{outcolor}{rgb}{0.545, 0.0, 0.0}



    
    % Prevent overflowing lines due to hard-to-break entities
    \sloppy 
    % Setup hyperref package
    \hypersetup{
      breaklinks=true,  % so long urls are correctly broken across lines
      colorlinks=true,
      urlcolor=urlcolor,
      linkcolor=linkcolor,
      citecolor=citecolor,
      }
    % Slightly bigger margins than the latex defaults
    
    \geometry{verbose,tmargin=1in,bmargin=1in,lmargin=1in,rmargin=1in}
    
    

    \begin{document}
    
    
    \maketitle
    
    

    
    \hypertarget{more-matrices}{%
\section{More Matrices}\label{more-matrices}}

This notebook continues your exploration of matrices.

\hypertarget{matrix-multiplication}{%
\subsection{Matrix Multiplication}\label{matrix-multiplication}}

Multiplying matrices is a little more complex than the operations we've
seen so far. There are two cases to consider, \emph{scalar
multiplication} (multiplying a matrix by a single number), and \emph{dot
product matrix multiplication} (multiplying a matrix by another matrix).

\hypertarget{scalar-multiplication}{%
\subsubsection{Scalar Multiplication}\label{scalar-multiplication}}

To multiply a matrix by a scalar value, you just multiply each element
by the scalar to produce a new matrix:

\begin{equation}2 \times \begin{bmatrix}1 & 2 & 3 \\4 & 5 & 6\end{bmatrix} = \begin{bmatrix}2 & 4 & 6 \\8 & 10 & 12\end{bmatrix}\end{equation}

In Python, you perform this calculation using the \textbf{*} operator:

    \begin{Verbatim}[commandchars=\\\{\}]
{\color{incolor}In [{\color{incolor}1}]:} \PY{k+kn}{import} \PY{n+nn}{numpy} \PY{k}{as} \PY{n+nn}{np}
        
        \PY{n}{A} \PY{o}{=} \PY{n}{np}\PY{o}{.}\PY{n}{array}\PY{p}{(}\PY{p}{[}\PY{p}{[}\PY{l+m+mi}{1}\PY{p}{,}\PY{l+m+mi}{2}\PY{p}{,}\PY{l+m+mi}{3}\PY{p}{]}\PY{p}{,}
                      \PY{p}{[}\PY{l+m+mi}{4}\PY{p}{,}\PY{l+m+mi}{5}\PY{p}{,}\PY{l+m+mi}{6}\PY{p}{]}\PY{p}{]}\PY{p}{)}
        \PY{n+nb}{print}\PY{p}{(}\PY{l+m+mi}{2} \PY{o}{*} \PY{n}{A}\PY{p}{)}
\end{Verbatim}


    \begin{Verbatim}[commandchars=\\\{\}]
[[ 2  4  6]
 [ 8 10 12]]

    \end{Verbatim}

    \hypertarget{dot-product-matrix-multiplication}{%
\subsubsection{Dot Product Matrix
Multiplication}\label{dot-product-matrix-multiplication}}

To mulitply two matrices together, you need to calculate the \emph{dot
product} of rows and columns. This means multiplying each of the
elements in each row of the first matrix by each of the elements in each
column of the second matrix and adding the results. We perform this
operation by applying the \emph{RC} rule - always multiplying
\textbf{\emph{R}}ows by \textbf{\emph{C}}olumns. For this to work, the
number of \textbf{\emph{columns}} in the first matrix must be the same
as the number of \textbf{\emph{rows}} in the second matrix so that the
matrices are \emph{conformable} for the dot product operation.

Sounds confusing, right?

Let's look at an example:

\begin{equation}\begin{bmatrix}1 & 2 & 3 \\4 & 5 & 6\end{bmatrix} \cdot \begin{bmatrix}9 & 8 \\ 7 & 6 \\ 5 & 4\end{bmatrix}\end{equation}

Note that the first matrix is 2x3, and the second matrix is 3x2. The
important thing here is that the first matrix has two rows, and the
second matrix has two columns. To perform the multiplication, we first
take the dot product of the first \textbf{\emph{row}} of the first
matrix (1,2,3) and the first \textbf{\emph{column}} of the second matrix
(9,7,5):

\begin{equation}(1,2,3) \cdot (9,7,5) = (1 \times 9) + (2 \times 7) + (3 \times 5) = 38\end{equation}

In our resulting matrix (which will always have the same number of
\textbf{\emph{rows}} as the first matrix, and the same number of
\textbf{\emph{columns}} as the second matrix), we can enter this into
the first row and first column element:

\begin{equation}\begin{bmatrix}38 & ?\\? & ?\end{bmatrix} \end{equation}

Now we can take the dot product of the first row of the first matrix and
the second column of the second matrix:

\begin{equation}(1,2,3) \cdot (8,6,4) = (1 \times 8) + (2 \times 6) + (3 \times 4) = 32\end{equation}

Let's add that to our resulting matrix in the first row and second
column element:

\begin{equation}\begin{bmatrix}38 & 32\\? & ?\end{bmatrix} \end{equation}

Now we can repeat this process for the second row of the first matrix
and the first column of the second matrix:

\begin{equation}(4,5,6) \cdot (9,7,5) = (4 \times 9) + (5 \times 7) + (6 \times 5) = 101\end{equation}

Which fills in the next element in the result:

\begin{equation}\begin{bmatrix}38 & 32\\101 & ?\end{bmatrix} \end{equation}

Finally, we get the dot product for the second row of the first matrix
and the second column of the second matrix:

\begin{equation}(4,5,6) \cdot (8,6,4) = (4 \times 8) + (5 \times 6) + (6 \times 4) = 86\end{equation}

Giving us:

\begin{equation}\begin{bmatrix}38 & 32\\101 & 86\end{bmatrix} \end{equation}

In Python, you can use the \emph{numpy.\textbf{dot}} function or the
**@** operator to multiply matrices and two-dimensional arrays:

    \begin{Verbatim}[commandchars=\\\{\}]
{\color{incolor}In [{\color{incolor}2}]:} \PY{k+kn}{import} \PY{n+nn}{numpy} \PY{k}{as} \PY{n+nn}{np}
        
        \PY{n}{A} \PY{o}{=} \PY{n}{np}\PY{o}{.}\PY{n}{array}\PY{p}{(}\PY{p}{[}\PY{p}{[}\PY{l+m+mi}{1}\PY{p}{,}\PY{l+m+mi}{2}\PY{p}{,}\PY{l+m+mi}{3}\PY{p}{]}\PY{p}{,}
                      \PY{p}{[}\PY{l+m+mi}{4}\PY{p}{,}\PY{l+m+mi}{5}\PY{p}{,}\PY{l+m+mi}{6}\PY{p}{]}\PY{p}{]}\PY{p}{)}
        \PY{n}{B} \PY{o}{=} \PY{n}{np}\PY{o}{.}\PY{n}{array}\PY{p}{(}\PY{p}{[}\PY{p}{[}\PY{l+m+mi}{9}\PY{p}{,}\PY{l+m+mi}{8}\PY{p}{]}\PY{p}{,}
                      \PY{p}{[}\PY{l+m+mi}{7}\PY{p}{,}\PY{l+m+mi}{6}\PY{p}{]}\PY{p}{,}
                      \PY{p}{[}\PY{l+m+mi}{5}\PY{p}{,}\PY{l+m+mi}{4}\PY{p}{]}\PY{p}{]}\PY{p}{)}
        \PY{n+nb}{print}\PY{p}{(}\PY{n}{np}\PY{o}{.}\PY{n}{dot}\PY{p}{(}\PY{n}{A}\PY{p}{,}\PY{n}{B}\PY{p}{)}\PY{p}{)}
        \PY{n+nb}{print}\PY{p}{(}\PY{n}{A} \PY{o}{@} \PY{n}{B}\PY{p}{)}
\end{Verbatim}


    \begin{Verbatim}[commandchars=\\\{\}]
[[ 38  32]
 [101  86]]
[[ 38  32]
 [101  86]]

    \end{Verbatim}

    This is one case where there is a difference in behavior between
\emph{numpy.\textbf{array}} and \emph{numpy.\textbf{matrix}}, You can
also use a regular multiplication (\textbf{*}) operator with a matrix,
but not with an array:

    \begin{Verbatim}[commandchars=\\\{\}]
{\color{incolor}In [{\color{incolor}3}]:} \PY{k+kn}{import} \PY{n+nn}{numpy} \PY{k}{as} \PY{n+nn}{np}
        
        \PY{n}{A} \PY{o}{=} \PY{n}{np}\PY{o}{.}\PY{n}{matrix}\PY{p}{(}\PY{p}{[}\PY{p}{[}\PY{l+m+mi}{1}\PY{p}{,}\PY{l+m+mi}{2}\PY{p}{,}\PY{l+m+mi}{3}\PY{p}{]}
                       \PY{p}{,}\PY{p}{[}\PY{l+m+mi}{4}\PY{p}{,}\PY{l+m+mi}{5}\PY{p}{,}\PY{l+m+mi}{6}\PY{p}{]}\PY{p}{]}\PY{p}{)}
        \PY{n}{B} \PY{o}{=} \PY{n}{np}\PY{o}{.}\PY{n}{matrix}\PY{p}{(}\PY{p}{[}\PY{p}{[}\PY{l+m+mi}{9}\PY{p}{,}\PY{l+m+mi}{8}\PY{p}{]}\PY{p}{,}
                       \PY{p}{[}\PY{l+m+mi}{7}\PY{p}{,}\PY{l+m+mi}{6}\PY{p}{]}\PY{p}{,}
                       \PY{p}{[}\PY{l+m+mi}{5}\PY{p}{,}\PY{l+m+mi}{4}\PY{p}{]}\PY{p}{]}\PY{p}{)}
        \PY{n+nb}{print}\PY{p}{(}\PY{n}{A} \PY{o}{*} \PY{n}{B}\PY{p}{)}
\end{Verbatim}


    \begin{Verbatim}[commandchars=\\\{\}]
[[ 38  32]
 [101  86]]

    \end{Verbatim}

    Note that, unlike with multiplication of regular scalar numbers, the
order of the operands in a multiplication operation is significant. For
scalar numbers, the \emph{commmutative law} of multiplication applies,
so for example:

\begin{equation}2 \times 4 = 4 \times 2\end{equation}

With matrix multiplication, things are different, for example:

\begin{equation}\begin{bmatrix}2 & 4 \\6 & 8\end{bmatrix} \cdot \begin{bmatrix}1 & 3 \\ 5 & 7\end{bmatrix} \ne \begin{bmatrix}1 & 3 \\ 5 & 7\end{bmatrix} \cdot \begin{bmatrix}2 & 4 \\6 & 8\end{bmatrix}\end{equation}

Run the following Python code to test this:

    \begin{Verbatim}[commandchars=\\\{\}]
{\color{incolor}In [{\color{incolor}4}]:} \PY{k+kn}{import} \PY{n+nn}{numpy} \PY{k}{as} \PY{n+nn}{np}
        
        \PY{n}{A} \PY{o}{=} \PY{n}{np}\PY{o}{.}\PY{n}{array}\PY{p}{(}\PY{p}{[}\PY{p}{[}\PY{l+m+mi}{2}\PY{p}{,}\PY{l+m+mi}{4}\PY{p}{]}\PY{p}{,}
                      \PY{p}{[}\PY{l+m+mi}{6}\PY{p}{,}\PY{l+m+mi}{8}\PY{p}{]}\PY{p}{]}\PY{p}{)}
        \PY{n}{B} \PY{o}{=} \PY{n}{np}\PY{o}{.}\PY{n}{array}\PY{p}{(}\PY{p}{[}\PY{p}{[}\PY{l+m+mi}{1}\PY{p}{,}\PY{l+m+mi}{3}\PY{p}{]}\PY{p}{,}
                      \PY{p}{[}\PY{l+m+mi}{5}\PY{p}{,}\PY{l+m+mi}{7}\PY{p}{]}\PY{p}{]}\PY{p}{)}
        \PY{n+nb}{print}\PY{p}{(}\PY{n}{A} \PY{o}{@} \PY{n}{B}\PY{p}{)}
        \PY{n+nb}{print}\PY{p}{(}\PY{n}{B} \PY{o}{@} \PY{n}{A}\PY{p}{)}
\end{Verbatim}


    \begin{Verbatim}[commandchars=\\\{\}]
[[22 34]
 [46 74]]
[[20 28]
 [52 76]]

    \end{Verbatim}

    \hypertarget{identity-matrices}{%
\subsection{Identity Matrices}\label{identity-matrices}}

An \emph{identity} matrix (usually indicated by a capital \textbf{I}) is
the equivalent in matrix terms of the number \textbf{1}. It always has
the same number of rows as columns, and it has the value \textbf{1} in
the diagonal element positions I1,1, I2,2, etc; and 0 in all other
element positions. Here's an example of a 3x3 identity matrix:

\begin{equation}\begin{bmatrix}1 & 0 & 0\\0 & 1 & 0\\0 & 0 & 1\end{bmatrix} \end{equation}

Multiplying any matrix by an identity matrix is the same as multiplying
a number by 1; the result is the same as the original value:

\begin{equation}\begin{bmatrix}1 & 2 & 3 \\4 & 5 & 6\\7 & 8 & 9\end{bmatrix} \cdot \begin{bmatrix}1 & 0 & 0\\0 & 1 & 0\\0 & 0 & 1\end{bmatrix} = \begin{bmatrix}1 & 2 & 3 \\4 & 5 & 6\\7 & 8 & 9\end{bmatrix} \end{equation}

If you doubt me, try the following Python code!

    \begin{Verbatim}[commandchars=\\\{\}]
{\color{incolor}In [{\color{incolor}5}]:} \PY{k+kn}{import} \PY{n+nn}{numpy} \PY{k}{as} \PY{n+nn}{np}
        
        \PY{n}{A} \PY{o}{=} \PY{n}{np}\PY{o}{.}\PY{n}{array}\PY{p}{(}\PY{p}{[}\PY{p}{[}\PY{l+m+mi}{1}\PY{p}{,}\PY{l+m+mi}{2}\PY{p}{,}\PY{l+m+mi}{3}\PY{p}{]}\PY{p}{,}
                      \PY{p}{[}\PY{l+m+mi}{4}\PY{p}{,}\PY{l+m+mi}{5}\PY{p}{,}\PY{l+m+mi}{6}\PY{p}{]}\PY{p}{,}
                      \PY{p}{[}\PY{l+m+mi}{7}\PY{p}{,}\PY{l+m+mi}{8}\PY{p}{,}\PY{l+m+mi}{9}\PY{p}{]}\PY{p}{]}\PY{p}{)}
        \PY{n}{B} \PY{o}{=} \PY{n}{np}\PY{o}{.}\PY{n}{array}\PY{p}{(}\PY{p}{[}\PY{p}{[}\PY{l+m+mi}{1}\PY{p}{,}\PY{l+m+mi}{0}\PY{p}{,}\PY{l+m+mi}{0}\PY{p}{]}\PY{p}{,}
                      \PY{p}{[}\PY{l+m+mi}{0}\PY{p}{,}\PY{l+m+mi}{1}\PY{p}{,}\PY{l+m+mi}{0}\PY{p}{]}\PY{p}{,}
                      \PY{p}{[}\PY{l+m+mi}{0}\PY{p}{,}\PY{l+m+mi}{0}\PY{p}{,}\PY{l+m+mi}{1}\PY{p}{]}\PY{p}{]}\PY{p}{)}
        \PY{n+nb}{print}\PY{p}{(}\PY{n}{A} \PY{o}{@} \PY{n}{B}\PY{p}{)}
\end{Verbatim}


    \begin{Verbatim}[commandchars=\\\{\}]
[[1 2 3]
 [4 5 6]
 [7 8 9]]

    \end{Verbatim}

    \hypertarget{matrix-division}{%
\subsection{Matrix Division}\label{matrix-division}}

You can't actually divide by a matrix; but when you want to divide
matrices, you can take advantage of the fact that division by a given
number is the same as multiplication by the reciprocal of that number.
For example:

\begin{equation}6 \div 3 = \frac{1}{3}\times 6 \end{equation}

In this case, 1/3 is the reciprocal of 3 (which as a fraction is 3/1 -
we ``flip'' the numerator and denominator to get the reciprocal). You
can also write 1/3 as 3-1.

\hypertarget{inverse-of-a-matrix}{%
\subsubsection{Inverse of a Matrix}\label{inverse-of-a-matrix}}

For matrix division, we use a related idea; we multiply by the
\emph{inverse} of a matrix:

\begin{equation}A \div B = A \cdot B^{-1}\end{equation}

The inverse of B is B-1 as long as the following equation is true:

\begin{equation}B \cdot B^{-1} = B^{-1} \cdot B = I\end{equation}

\textbf{I}, you may recall, is an \emph{identity} matrix; the matrix
equivalent of 1.

So how do you calculate the inverse of a matrix? For a 2x2 matrix, you
can follow this formula:

\begin{equation}\begin{bmatrix}a & b\\c & d\end{bmatrix}^{-1} = \frac{1}{ad-bc}  \begin{bmatrix}d & -b\\-c & a\end{bmatrix}\end{equation}

What happened there? - We swapped the positions of \emph{a} and \emph{d}
- We changed the signs of \emph{b} and \emph{c} - We multiplied the
resulting matrix by 1 over the \emph{determinant} of the matrix
(\emph{ad-bc})

Let's try with some actual numbers:

\begin{equation}\begin{bmatrix}6 & 2\\1 & 2\end{bmatrix}^{-1} = \frac{1}{(6\times2)-(2\times1)}  \begin{bmatrix}2 & -2\\-1 & 6\end{bmatrix}\end{equation}

So:

\begin{equation}\begin{bmatrix}6 & 2\\1 & 2\end{bmatrix}^{-1} = \frac{1}{10}  \begin{bmatrix}2 & -2\\-1 & 6\end{bmatrix}\end{equation}

Which gives us the result:

\begin{equation}\begin{bmatrix}6 & 2\\1 & 2\end{bmatrix}^{-1} = \begin{bmatrix}0.2 & -0.2\\-0.1 & 0.6\end{bmatrix}\end{equation}

To check this, we can multiply the original matrix by its inverse to see
if we get an identity matrix. This makes sense if you think about it; in
the same way that 3 x 1/3 = 1, a matrix multiplied by its inverse
results in an identity matrix:

\begin{equation}\begin{bmatrix}6 & 2\\1 & 2\end{bmatrix} \cdot \begin{bmatrix}0.2 & -0.2\\-0.1 & 0.6\end{bmatrix} = \begin{bmatrix}(6\times0.2)+(2\times-0.1) & (6\times-0.2)+(2\times0.6)\\(1\times0.2)+(2\times-0.1) & (1\times-0.2)+(2\times0.6)\end{bmatrix} = \begin{bmatrix}1 & 0\\0 & 1\end{bmatrix}\end{equation}

Note that not every matrix has an inverse - for example, if the
determinant works out to be 0, the inverse matrix is not defined.

In Python, you can use the \emph{numpy.linalg.\textbf{inv}} function to
get the inverse of a matrix in an \emph{array} or \emph{matrix} object:

    \begin{Verbatim}[commandchars=\\\{\}]
{\color{incolor}In [{\color{incolor}6}]:} \PY{k+kn}{import} \PY{n+nn}{numpy} \PY{k}{as} \PY{n+nn}{np}
        
        \PY{n}{B} \PY{o}{=} \PY{n}{np}\PY{o}{.}\PY{n}{array}\PY{p}{(}\PY{p}{[}\PY{p}{[}\PY{l+m+mi}{6}\PY{p}{,}\PY{l+m+mi}{2}\PY{p}{]}\PY{p}{,}
                      \PY{p}{[}\PY{l+m+mi}{1}\PY{p}{,}\PY{l+m+mi}{2}\PY{p}{]}\PY{p}{]}\PY{p}{)}
        
        \PY{n+nb}{print}\PY{p}{(}\PY{n}{np}\PY{o}{.}\PY{n}{linalg}\PY{o}{.}\PY{n}{inv}\PY{p}{(}\PY{n}{B}\PY{p}{)}\PY{p}{)}
\end{Verbatim}


    \begin{Verbatim}[commandchars=\\\{\}]
[[ 0.2 -0.2]
 [-0.1  0.6]]

    \end{Verbatim}

    Additionally, the \emph{matrix} type has an \textbf{\emph{I}} method
that returns the inverse matrix:

    \begin{Verbatim}[commandchars=\\\{\}]
{\color{incolor}In [{\color{incolor}7}]:} \PY{k+kn}{import} \PY{n+nn}{numpy} \PY{k}{as} \PY{n+nn}{np}
        
        \PY{n}{B} \PY{o}{=} \PY{n}{np}\PY{o}{.}\PY{n}{matrix}\PY{p}{(}\PY{p}{[}\PY{p}{[}\PY{l+m+mi}{6}\PY{p}{,}\PY{l+m+mi}{2}\PY{p}{]}\PY{p}{,}
                      \PY{p}{[}\PY{l+m+mi}{1}\PY{p}{,}\PY{l+m+mi}{2}\PY{p}{]}\PY{p}{]}\PY{p}{)}
        
        \PY{n+nb}{print}\PY{p}{(}\PY{n}{B}\PY{o}{.}\PY{n}{I}\PY{p}{)}
\end{Verbatim}


    \begin{Verbatim}[commandchars=\\\{\}]
[[ 0.2 -0.2]
 [-0.1  0.6]]

    \end{Verbatim}

    For larger matrices, the process to calculate the inverse is more
complex. Let's explore an example based on the following matrix:

\begin{equation}\begin{bmatrix}4 & 2 & 2\\6 & 2 & 4\\2 & 2 & 8\end{bmatrix} \end{equation}

The process to find the inverse consists of the following steps:

1: Create a matrix of \emph{minors} by calculating the
\emph{determinant} for each element in the matrix based on the elements
that are not in the same row or column; like this:

\begin{equation}\begin{bmatrix}\color{blue}4 & \color{lightgray}2 & \color{lightgray}2\\\color{lightgray}6 & \color{red}2 & \color{red}4\\\color{lightgray}2 & \color{red}2 & \color{red}8\end{bmatrix}\;\;\;\;(2\times8) - (4\times2) = 8\;\;\;\;\begin{bmatrix}8 & \color{lightgray}? & \color{lightgray}?\\\color{lightgray}? & \color{lightgray}? & \color{lightgray}?\\\color{lightgray}? & \color{lightgray}? & \color{lightgray}?\end{bmatrix} \end{equation}

\begin{equation}\begin{bmatrix}\color{lightgray}4 & \color{blue}2 & \color{lightgray}2\\\color{red}6 & \color{lightgray}2 & \color{red}4\\\color{red}2 & \color{lightgray}2 & \color{red}8\end{bmatrix}\;\;\;\;(6\times8) - (4\times2) = 40\;\;\;\;\begin{bmatrix}8 & 40 & \color{lightgray}?\\\color{lightgray}? & \color{lightgray}? & \color{lightgray}?\\\color{lightgray}? & \color{lightgray}? & \color{lightgray}?\end{bmatrix}\end{equation}

\begin{equation}\begin{bmatrix}\color{lightgray}4 & \color{lightgray}2 & \color{blue}2\\\color{red}6 & \color{red}2 & \color{lightgray}4\\\color{red}2 & \color{red}2 & \color{lightgray}8\end{bmatrix}\;\;\;\;(6\times2) - (2\times2) = 8\;\;\;\;\begin{bmatrix}8 & 40 & 8\\\color{lightgray}? & \color{lightgray}? & \color{lightgray}?\\\color{lightgray}? & \color{lightgray}? & \color{lightgray}?\end{bmatrix} \end{equation}

\begin{equation}\begin{bmatrix}\color{lightgray}4 & \color{red}2 & \color{red}2\\\color{blue}6 & \color{lightgray}2 & \color{lightgray}4\\\color{lightgray}2 & \color{red}2 & \color{red}8\end{bmatrix}\;\;\;\;(2\times8) - (2\times2) = 12\;\;\;\;\begin{bmatrix}8 & 40 & 8\\12 & \color{lightgray}? & \color{lightgray}?\\\color{lightgray}? & \color{lightgray}? & \color{lightgray}?\end{bmatrix} \end{equation}

\begin{equation}\begin{bmatrix}\color{red}4 & \color{lightgray}2 & \color{red}2\\\color{lightgray}6 & \color{blue}2 & \color{lightgray}4\\\color{red}2 & \color{lightgray}2 & \color{red}8\end{bmatrix}\;\;\;\;(4\times8) - (2\times2) = 28\;\;\;\;\begin{bmatrix}8 & 40 & 8\\12 & 28 & \color{lightgray}?\\\color{lightgray}? & \color{lightgray}? & \color{lightgray}?\end{bmatrix} \end{equation}

\begin{equation}\begin{bmatrix}\color{red}4 & \color{red}2 & \color{lightgray}2\\\color{lightgray}6 & \color{lightgray}2 & \color{blue}4\\\color{red}2 & \color{red}2 & \color{lightgray}8\end{bmatrix}\;\;\;\;(4\times2) - (2\times2) = 4\;\;\;\;\begin{bmatrix}8 & 40 & 8\\12 & 28 & 4\\\color{lightgray}? & \color{lightgray}? & \color{lightgray}?\end{bmatrix} \end{equation}

\begin{equation}\begin{bmatrix}\color{lightgray}4 & \color{red}2 & \color{red}2\\\color{lightgray}6 & \color{red}2 & \color{red}4\\\color{blue}2 & \color{lightgray}2 & \color{lightgray}8\end{bmatrix}\;\;\;\;(2\times4) - (2\times2) = 4\;\;\;\;\begin{bmatrix}8 & 40 & 8\\12 & 28 & 4\\4 & \color{lightgray}? & \color{lightgray}?\end{bmatrix} \end{equation}

\begin{equation}\begin{bmatrix}\color{red}4 & \color{lightgray}2 & \color{red}2\\\color{red}6 & \color{lightgray}2 & \color{red}4\\\color{lightgray}2 & \color{blue}2 & \color{lightgray}8\end{bmatrix}\;\;\;\;(4\times4) - (2\times6) = 4\;\;\;\;\begin{bmatrix}8 & 40 & 8\\12 & 28 & 4\\4 & 4 & \color{lightgray}?\end{bmatrix} \end{equation}

\begin{equation}\begin{bmatrix}\color{red}4 & \color{red}2 & \color{lightgray}2\\\color{red}6 & \color{red}2 & \color{lightgray}4\\\color{lightgray}2 & \color{lightgray}2 & \color{blue}8\end{bmatrix}\;\;\;\;(4\times2) - (2\times6) = -4\;\;\;\;\begin{bmatrix}8 & 40 & 8\\12 & 28 & 4\\4 & 4 & -4\end{bmatrix} \end{equation}

2: Apply \emph{cofactors} to the matrix by switching the sign of every
alternate element in the matrix of minors:

\begin{equation}\begin{bmatrix}8 & -40 & 8\\-12 & 28 & -4\\4 & -4 & -4\end{bmatrix} \end{equation}

3: \emph{Adjugate} by transposing elements diagonally:

\begin{equation}\begin{bmatrix}8 & \color{green}-\color{green}1\color{green}2 & \color{orange}4\\\color{green}-\color{green}4\color{green}0 & 28 & \color{purple}-\color{purple}4\\\color{orange}8 & \color{purple}-\color{purple}4 & -4\end{bmatrix} \end{equation}

4: Multiply by 1/determinant of the original matrix. To find this,
multiply each of the top row elements by their corresponding minor
determinants (which we calculated earlier in the matrix of minors), and
then subtract the second from the first and add the third:

\begin{equation}Determinant = (4 \times 8) - (2 \times 40) + (2 \times 8) = -32\end{equation}

\begin{equation}\frac{1}{-32}\begin{bmatrix}8 & -12 & 4\\-40 & 28 & -4\\8 & -4 & -4\end{bmatrix} =  \begin{bmatrix}-0.25 & 0.375 & -0.125\\1.25 & -0.875 & 0.125\\-0.25 & 0.125 & 0.125\end{bmatrix}\end{equation}

Let's verify that the original matrix multiplied by the inverse results
in an identity matrix:

\begin{equation}\begin{bmatrix}4 & 2 & 2\\6 & 2 & 4\\2 & 2 & 8\end{bmatrix} \cdot \begin{bmatrix}-0.25 & 0.375 & -0.125\\1.25 & -0.875 & 0.125\\-0.25 & 0.125 & 0.125\end{bmatrix}\end{equation}

\begin{equation}= \begin{bmatrix}(4\times-0.25)+(2\times1.25)+(2\times-0.25) & (4\times0.375)+(2\times-0.875)+(2\times0.125) & (4\times-0.125)+(2\times-0.125)+(2\times0.125)\\(6\times-0.25)+(2\times1.25)+(4\times-0.25) & (6\times0.375)+(2\times-0.875)+(4\times0.125) & (6\times-0.125)+(2\times-0.125)+(4\times0.125)\\(2\times-0.25)+(2\times1.25)+(8\times-0.25) & (2\times0.375)+(2\times-0.875)+(8\times0.125) & (2\times-0.125)+(2\times-0.125)+(8\times0.125)\end{bmatrix} \end{equation}

\begin{equation}= \begin{bmatrix}1 & 0 & 0\\0 & 1 & 0\\0 & 0 & 1\end{bmatrix} \end{equation}

As you can see, this can get pretty complicated - which is why we
usually use a calculator or a computer program. You can run the
following Python code to verify that the inverse matrix we calculated is
correct:

    \begin{Verbatim}[commandchars=\\\{\}]
{\color{incolor}In [{\color{incolor}8}]:} \PY{k+kn}{import} \PY{n+nn}{numpy} \PY{k}{as} \PY{n+nn}{np}
        
        \PY{n}{B} \PY{o}{=} \PY{n}{np}\PY{o}{.}\PY{n}{array}\PY{p}{(}\PY{p}{[}\PY{p}{[}\PY{l+m+mi}{4}\PY{p}{,}\PY{l+m+mi}{2}\PY{p}{,}\PY{l+m+mi}{2}\PY{p}{]}\PY{p}{,}
                      \PY{p}{[}\PY{l+m+mi}{6}\PY{p}{,}\PY{l+m+mi}{2}\PY{p}{,}\PY{l+m+mi}{4}\PY{p}{]}\PY{p}{,}
                      \PY{p}{[}\PY{l+m+mi}{2}\PY{p}{,}\PY{l+m+mi}{2}\PY{p}{,}\PY{l+m+mi}{8}\PY{p}{]}\PY{p}{]}\PY{p}{)}
        
        \PY{n+nb}{print}\PY{p}{(}\PY{n}{np}\PY{o}{.}\PY{n}{linalg}\PY{o}{.}\PY{n}{inv}\PY{p}{(}\PY{n}{B}\PY{p}{)}\PY{p}{)}
\end{Verbatim}


    \begin{Verbatim}[commandchars=\\\{\}]
[[-0.25   0.375 -0.125]
 [ 1.25  -0.875  0.125]
 [-0.25   0.125  0.125]]

    \end{Verbatim}

    \hypertarget{multiplying-by-an-inverse-matrix}{%
\subsubsection{Multiplying by an Inverse
Matrix}\label{multiplying-by-an-inverse-matrix}}

Now that you know how to calculate an inverse matrix, you can use that
knowledge to multiply the inverse of a matrix by another matrix as an
alternative to division:

\begin{equation}\begin{bmatrix}1 & 2\\3 & 4\end{bmatrix} \cdot \begin{bmatrix}6 & 2\\1 & 2\end{bmatrix}^{-1} \end{equation}

\begin{equation}=\begin{bmatrix}1 & 2\\3 & 4\end{bmatrix} \cdot \begin{bmatrix}0.2 & -0.2\\-0.1 & 0.6\end{bmatrix}  \end{equation}

\begin{equation}=\begin{bmatrix}(1\times0.2)+(2\times-0.1) & (1\times-0.2)+(2\times0.6)\\(3\times0.2)+(4\times-0.1) & (3\times-0.2)+(4\times0.6)\end{bmatrix}\end{equation}

\begin{equation}=\begin{bmatrix}0 & 1\\0.2 & 1.8\end{bmatrix}\end{equation}

Here's the Python code to calculate this:

    \begin{Verbatim}[commandchars=\\\{\}]
{\color{incolor}In [{\color{incolor}9}]:} \PY{k+kn}{import} \PY{n+nn}{numpy} \PY{k}{as} \PY{n+nn}{np}
        
        \PY{n}{A} \PY{o}{=} \PY{n}{np}\PY{o}{.}\PY{n}{array}\PY{p}{(}\PY{p}{[}\PY{p}{[}\PY{l+m+mi}{1}\PY{p}{,}\PY{l+m+mi}{2}\PY{p}{]}\PY{p}{,}
                      \PY{p}{[}\PY{l+m+mi}{3}\PY{p}{,}\PY{l+m+mi}{4}\PY{p}{]}\PY{p}{]}\PY{p}{)}
        
        \PY{n}{B} \PY{o}{=} \PY{n}{np}\PY{o}{.}\PY{n}{array}\PY{p}{(}\PY{p}{[}\PY{p}{[}\PY{l+m+mi}{6}\PY{p}{,}\PY{l+m+mi}{2}\PY{p}{]}\PY{p}{,}
                      \PY{p}{[}\PY{l+m+mi}{1}\PY{p}{,}\PY{l+m+mi}{2}\PY{p}{]}\PY{p}{]}\PY{p}{)}
        
        
        \PY{n}{C} \PY{o}{=} \PY{n}{A} \PY{o}{@} \PY{n}{np}\PY{o}{.}\PY{n}{linalg}\PY{o}{.}\PY{n}{inv}\PY{p}{(}\PY{n}{B}\PY{p}{)}
        
        \PY{n+nb}{print}\PY{p}{(}\PY{n}{C}\PY{p}{)}
\end{Verbatim}


    \begin{Verbatim}[commandchars=\\\{\}]
[[0.  1. ]
 [0.2 1.8]]

    \end{Verbatim}

    \hypertarget{solving-systems-of-equations-with-matrices}{%
\subsection{Solving Systems of Equations with
Matrices}\label{solving-systems-of-equations-with-matrices}}

One of the great things about matrices, is that they can help us solve
systems of equations. For example, consider the following system of
equations:

\begin{equation}2x + 4y = 18\end{equation}
\begin{equation}6x + 2y = 34\end{equation}

We can write this in matrix form, like this:

\begin{equation}\begin{bmatrix}2 & 4\\6 & 2\end{bmatrix} \cdot \begin{bmatrix}x\\y\end{bmatrix}=\begin{bmatrix}18\\34\end{bmatrix}\end{equation}

Note that the variables (\textbf{\emph{x}} and \textbf{\emph{y}}) are
arranged as a column in one matrix, which is multiplied by a matrix
containing the coefficients to produce as matrix containing the results.
If you calculate the dot product on the left side, you can see clearly
that this represents the original equations:

\begin{equation}\begin{bmatrix}2x + 4y\\6x + 2y\end{bmatrix} =\begin{bmatrix}18\\34\end{bmatrix}\end{equation}

Now. let's name our matrices so we can better understand what comes
next:

\begin{equation}A=\begin{bmatrix}2 & 4\\6 & 2\end{bmatrix}\;\;\;\;X=\begin{bmatrix}x\\y\end{bmatrix}\;\;\;\;B=\begin{bmatrix}18\\34\end{bmatrix}\end{equation}

We already know that \textbf{\emph{A • X = B}}, which arithmetically
means that \textbf{\emph{X = B ÷ A}}. Since we can't actually divide by
a matrix, we need to multiply by the inverse; so we can find the values
for our variables (\emph{X}) like this: \textbf{\emph{X = A-1 • B}}

So, first we need the inverse of A:

\begin{equation}\begin{bmatrix}2 & 4\\6 & 2\end{bmatrix}^{-1} = \frac{1}{(2\times2)-(4\times6)}  \begin{bmatrix}2 & -4\\-6 & 2\end{bmatrix}\end{equation}

\begin{equation}= \frac{1}{-20}  \begin{bmatrix}2 & -4\\-6 & 2\end{bmatrix}\end{equation}

\begin{equation}=\begin{bmatrix}-0.1 & 0.2\\0.3 & -0.1\end{bmatrix}\end{equation}

Then we just multiply this with B:

\begin{equation}X = \begin{bmatrix}-0.1 & 0.2\\0.3 & -0.1\end{bmatrix} \cdot \begin{bmatrix}18\\34\end{bmatrix}\end{equation}

\begin{equation}X = \begin{bmatrix}(-0.1 \times 18)+(0.2 \times 34)\\(0.3\times18)+(-0.1\times34)\end{bmatrix}\end{equation}

\begin{equation}X = \begin{bmatrix}5\\2\end{bmatrix}\end{equation}

The resulting matrix (\emph{X}) contains the values for our \emph{x} and
\emph{y} variables, and we can check these by plugging them into the
original equations:

\begin{equation}(2\times5) + (4\times2) = 18\end{equation}
\begin{equation}(6\times5) + (2\times2) = 34\end{equation}

These of course simplify to:

\begin{equation}10 + 8 = 18\end{equation}
\begin{equation}30 + 4 = 34\end{equation}

So our variable values are correct.

Here's the Python code to do all of this:

    \begin{Verbatim}[commandchars=\\\{\}]
{\color{incolor}In [{\color{incolor}10}]:} \PY{k+kn}{import} \PY{n+nn}{numpy} \PY{k}{as} \PY{n+nn}{np}
         
         \PY{n}{A} \PY{o}{=} \PY{n}{np}\PY{o}{.}\PY{n}{array}\PY{p}{(}\PY{p}{[}\PY{p}{[}\PY{l+m+mi}{2}\PY{p}{,}\PY{l+m+mi}{4}\PY{p}{]}\PY{p}{,}
                       \PY{p}{[}\PY{l+m+mi}{6}\PY{p}{,}\PY{l+m+mi}{2}\PY{p}{]}\PY{p}{]}\PY{p}{)}
         
         \PY{n}{B} \PY{o}{=} \PY{n}{np}\PY{o}{.}\PY{n}{array}\PY{p}{(}\PY{p}{[}\PY{p}{[}\PY{l+m+mi}{18}\PY{p}{]}\PY{p}{,}
                       \PY{p}{[}\PY{l+m+mi}{34}\PY{p}{]}\PY{p}{]}\PY{p}{)}
         
         \PY{n}{C} \PY{o}{=} \PY{n}{np}\PY{o}{.}\PY{n}{linalg}\PY{o}{.}\PY{n}{inv}\PY{p}{(}\PY{n}{A}\PY{p}{)} \PY{o}{@} \PY{n}{B}
         
         \PY{n+nb}{print}\PY{p}{(}\PY{n}{C}\PY{p}{)}
\end{Verbatim}


    \begin{Verbatim}[commandchars=\\\{\}]
[[5.]
 [2.]]

    \end{Verbatim}


    % Add a bibliography block to the postdoc
    
    
    
    \end{document}
